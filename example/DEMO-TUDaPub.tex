\documentclass[
	ngerman,
	accentcolor=9c,% Farbe für Hervorhebungen auf Basis der Deklarationen in den Corporate Design Richtlinien
%	logofile=example-image, %Falls die Logo Dateien nicht vorliegen
	]{tudapub}

\usepackage[english, main=ngerman]{babel}
\usepackage[autostyle]{csquotes}

\usepackage{biblatex}
\bibliography{DEMO-TUDaBibliography}

%Formatierungen für Beispiele in diesem Dokument. Im Allgemeinen nicht notwendig!
\let\file\texttt
\let\code\texttt
\let\pck\textsf
\let\cls\textsf

\usepackage{hologo}

\begin{document}

%Zusätzliche Metadaten für PDF/A. In diesem Fall notwendig, weil Titel ein Makro enthält.
\Metadata{
	author=Marei Peischl (peiTeX),
	title=TUDaPub - LaTeX-Paper im Corporate Design der TU Darmstadt,
	subject=Basisdokumentation und Template zur Nutzung der tudapub-Dokumentenkasse,
	date=2019-04-29,
	keywords=TU Darmstadt \sep Corporate Design \sep LaTeX
}

\title{TUDaPub -- \LaTeX-Paper im Corporate Design der TU Darmstadt}
\subtitle{Die Dokumentenklasse tudapub}
\author{Marei Peischl\thanks{pei\TeX{} \TeX{}nical Solutions}\and der \TeX-Löwe}

\titlegraphic{
%	%Folgende Box kann selbstverständlich durch ein mit \includegraphics geladenes Bild ersetzt werden.
	\color{black!30}\rule{\width}{\height}
}


%Varianten der Infoboxen
\addTitleBox{Institutsbezeichnung bzw. Logo in 2/3 Größe, vgl. \code{addTitleBoxLogo}}
%\addTitleBoxLogo{example-image}
%\addTitleBoxLogo*{\includegraphics[width=.3\linewidth]{example-image}}



\maketitle

\begin{abstract}
	Dieses Dokument stellt ein Template und gleichzeitig die Verwendungsdokumentation zur Dokumentenklasse tudapub, einem Teil des TUDa-CI-Paketes dar.

	Noch befindet sich das Projekt in der Testphase. Sollten Sie Fragen, Wünsche und Anregungen haben, bitten wir um eine entsprechende Mitteilung.
\end{abstract}

\begin{abstract}[english]
	Example for an additional abstract in English.
\end{abstract}



\tableofcontents

\section{Über diese Datei}
Die Datei \file{DEMO-TUDaPub.tex} beziehungsweise ihre Ausgabe \file{DEMO-TUDaPub.pdf} ist die Dokumentation der Dokumentenklasse \file{tudapub.sty}.

Sie ist Teil des TUDa-CI-Bundles und basiert in Teilen auf dem tuddesign-Paket von C.~v.~Loewenich und J.~Werner.

In diesem Dokument werden die speziellen Optionen und Einstellungsmöglichkeiten erläutert.

\section{Verwendung}
Die Klasse wird wie gewohnt geladen:
\begin{verbatim}
\documentclass[<Optionen>]{tudapub}
\end{verbatim}
Im folgenden werden die möglichen Optionen beschrieben.

\subsection{Klassenoptionen}

\begin{description}
	\item[class=<article|report|book>] Diese Option legt die Basisdokumentenklasse fest. Die Werte laden die entsprechende KOMA-Script-Klasse \cite{scrguide}. Der Wert \code{article} lädt somit die Klasse \code{scrartcl}.

	KOMA-Script ist eine Sammlung von Klassen und Paketen für \LaTeX, die neben den typografischen Anpassungen an den Europäischen Raum auch die Konfigurationsmöglichkeiten stark erweitert.
	\item[color=<Farbe>] Wählt die Schmuckfarbe für die Nutzung in der Identitätsleiste aus. Die Farbcodes finden sich in der Farbübersicht in den Corporate Design Richtlinien. Neben diesen Farben kann prinzipiell jede beliebige Farbe übergeben werden. Die Optionen \code{accentcolor}, \code{textaccentcolor} und \code{identbarcolor} werden anlog direkt an \pck{tudacolors} übergeben. Auf diesem Weg können die Farben unabhängig voneinander gesetzt werden.
	\item[marginpar] Schaltet die Randnotizspalte um. Voreingestellt ist \code{auto}. Dies bedeutet, dass die Randnotizspalte wie im Corporate Design Handbuch \cite{TUDaGuideline} über die fünfte Spalte läuft.  \marginpar{Beispiel für eine Randnotiz}.
	Darüber werden auch die Werte \code{true} und \code{false} akzeptiert. \code{false} setzt die Breite der Randnotizspalte auf 0. Der Mechanismus selbst wird nicht deaktiviert.

	Randnotizen werden über den komafont-Mechanismus \cite[vgl.][]{scrguide} im Element \code{marginpar} gesetzt. Seine Voreinstellung entspricht
	\begin{verbatim}
	\setkomafont{marginpar}{\accentfont}
	\end{verbatim}
	Um zusätzlich farbige Randnotizen zu setzen, könnte dies geändert werden, über
	\begin{verbatim}
	\addtokomafont{marginpar}{\color{textaccentcolor}}
	\end{verbatim}
	\marginline{
		\includegraphics[width=\marginparwidth]{example-image}\\
		Flattersatz in der marginnline aus \KOMAScript
	}
	\item[twocolumn] Aktiviert den zweispaltigen Modus global. In diesem Fall werden jedoch aufgrund ihrer Natur zwei Randnotizspalten erzeugt. Eine Nutzung in Kombination mit \code{marginpar=auto} ist daher in den meisten Fällen fragwürdig. Falls der zweispaltige Modus lediglich lokal aktiviert wird, entfällt dieses Verhalten, allerdings werden dann Randnotizen deaktiviert.
	\item[ruledheaders] Wählt den Stil der Überschriften aus. \code{ruledheaders=all} wählt den mit Linien eingerahmten Stil für alle bis zur \verb+\subsubsection+. Bei \code{chapter} beziehungsweise \code{section} ist dieser Stil entsprechend beschränkt. False lädt den Standardstil aus \KOMAScript.
	\item[title=default/small/large] Die relativ große Schriftgröße des Titels kann – insbesondere bei langen Titel für Abschlussarbeiten – zu Platzproblemen führen. Die Option \code{title} verhindert dies, indem die für das jeweils kleinere Papierformat eingestellten Schriftgrößen geladen werden. Die Voreinstellung ist \code{default} und entspricht dem Wert \code{large}.
  \item[type] Als Typ stehen im Moment \code{publication} und \code{thesis} zur Verfügung. Die besonderen Möglichkeiten im Typ thesis sind in der Datei DEMO-TUDaThesis.tex/.pdf geschildert. Voreingestellt ist \code{publication}.
	Zusätzlich existiert ab Version 1.2 noch der Modus \code{intern}. Dieser wählt die Optionen \code{titlepage=false} für einen Titelkopf statt Titelseiten, sowie die TUDaPub-Optionen \code{pdfa=""false} und \code{IMRAD=false}. Dieser Modus ist für kurze, interne Berichte gedacht.
	\item[headline] Die Kopfzeile verfügt über die im Corporate Design beschriebenen Layoutmöglichkeiten über den Wert \code{automark}. Da diese Lösung typografisch nicht sonderlich sinnvoll ist, ist es auch möglich, diese abzuschalten. Voreingestellt ist ein Stil ohne Kolumnentitel.
	\item[logo] Option für die Titelseite, siehe \ref{sec:title}
	\item[colorback] Option für die Titelseite, siehe \ref{sec:title}
	\item[IMRAD=true/false] Deaktiviert die Prüfung auf IMRAD-Labels, siehe Abschnitt \ref{sec:IMRAD}.
	\item[logofile=<Dateipfad>] Erlaubt es ein alternatives Logo zu übergeben. Diese Option existiert, um die Templates auch ohne das TUDa-Logo nutzen zu können. Die Logos sind der internen Verwendung vorbehalten und dürfen daher nicht mit diesem Template veröffentlicht werden.
\end{description}

\noindent Der Rest der Dokumentenklasse entspricht dem Standard von \KOMAScript, vgl. Abschnitt \ref{sec:KOMA}.

\subsection{Die Titelseite}
\label{sec:title}

Die Titelseite wird von tudapub automatisch generiert. Die Verwendung hierfür entspricht größtenteils der klassischen Methode unter Verwendung von \code{maketitle}.

Die \KOMAScript-Option \code{titlepage} erlaubt es üblicherweise, zwischen Titelseiten und Titelkopf umzuschalten. Bis Version 1.2 war diese Option deaktiviert. Mittlerweile existiert ein Modus für einen Titelkopf, dieser entspricht jedoch nicht den offiziellen Vorgaben und ist für interne Verwendung gedacht. Aufgrund der Implementierung wird \code{titlepage=true} identisch zu \code{titlepage=firstiscover} behandelt.

Die Makros wie \code{author}, \code{title}, \code{subtitle} und \code{date} entsprechen der normalen Verwendung. Für die Institutszuweisung kann ebenfalls wie gewohnt \code{thanks} verwendet werden.

Zusätzlich stehen die Makros \code{titlegraphic} und \code{addTitleBox} zur Verfügung um weitere Daten zu übergeben.

\minisec{titlegraphic}
Das Makro \code{titlegraphic} akzeptiert beliebigen Inhalt. Dieser wird bündig mit der oberen Ecke im Hauptteil der Titelseite platziert.
Üblicherweise wird dieses Makro zur Platzierung einer Grafik genutzt:

\begin{verbatim}
\titlegraphic{\includegraphics[width=\width]{example-image}}
\end{verbatim}

Um die Größe des Bildes passend zu wählen können die Makros \code{width}/\code{height} verwendet werden. Darüber hinaus existiert ab Version 3.19 auch die Variante \code{titlegraphic*} mit der die Skalierung und ein ggf. notwendiger Beschnitt für das Füllen des reservierten Bereichs automatisch geschieht.

Zusätzliche Logos und Informationen können über Boxen ergänzt werden.

\begin{verbatim}
\addTitleBox{Institut 1}
\end{verbatim}
Die Institutsboxen werden mit vorgegebenem Abstand unter dem Logo platziert. Hier kann Text erscheinen oder auch ein Institutslogo. Der Hintergrund ist weiß.

Um die Institutsboxen für Logos zu verwenden, liefert \cls{tudapub} das Makro \code{addTitleBoxLogo}. Als Argument akzeptiert es einen Bilddateipfad.

\begin{verbatim}
\addTitleBoxLogo{example-image}
\addTitleBoxLogo*{\includegraphics[width=\linewidth]{example-image}}
\end{verbatim}

\subsubsection{Sponsorenlogos}
Ab Version 3.0 steht in TUDaPub auch der Sponsorenmechanismus der TUDaLeaflet-Klasse zur Verfügung. Damit ist es Möglich Sponsorenlogos unterhalb des Titelbildes zu platzieren.

Sponsorenlogos werden üblicherweise über
\begin{verbatim}
	\AddSponsor{<logo1>}
	\AddSponsor{<log2>}
\end{verbatim}
übergeben. Innerhalb des Arguments ist \verb+\height+ so gesetzt. Somit werden im Beispiel alle Logos auf die gleiche Höhe gesetzt. Der Abstand dazwischen wird entsprechend aufgefüllt, sodass der gesamte Block immer links und rechts mit dem Text abschließt.

Die zweite Variante ermöglicht die Platzierung mit manueller vertikaler Ausrichtung, wie es bei Logos mit unterschiedlicher Höhe notwendig sein könnte. Hierbei werden lediglich die Abstände und Trennlinien um die Logos ergänzt:

\begin{verbatim}
	\sponsors{
		<logo1><logo2>
	}
\end{verbatim}


\subsubsection{Optionen für die Titelseite}
Die Position des Logos ist umschaltbar. Dies geschieht über die Dokumentenklassenoption \code{logo=head/""body}.
\begin{description}
	\item[logo=head] Das Logo wird im Kopf direkt neben dem Titel platziert. Der Titel wird in der Breite reduziert. Der Hintergrund des Titels wird in der Farbe der Identitätsleiste eingefärbt. Zusätzliche Infoboxen (s.u.) werden ebenfalls im Kopf platziert.
	\item[logo=body] Das Logo samt der Infoboxen wird im Körper der Titelseite platziert.
\end{description}

Darüber hinaus lässt sich die Farbgebung umschalten.
Die Option \code{colorback} schaltet zwischen dem farbigen Block auf der Titelseite und weißem Hintergrund um. Sie ist in der Voreinstellung aktiviert, sodass ein Farbiger Block erzeugt wird.

Mit Version 3.08 werden zusätzlich zu den \code{true}/\code{false} Werten auch die Werte \code{title}, \code{head} und \code{body} für die Option colorback ergänzt.
Mit diesen ist es möglich nach der Auswahl der Logoposition eine Korrektur an der Farbgebung durchzuführen. Alle drei neuen Werte aktivieren die Hintergrundfarbe und schalten deren Position um:
\begin{description}
	\item[colorback=true/false] Aktiviert/Deaktiviert die Farbfläche auf der Titelseite. Dieser Schalter ist unabhängig von der aktuelle Position der farbigen Fläche. Die übrigen Optionen aktivieren zusätzlich auch die Ausgabe der Farbfläche, falls diese zuvor deaktiviert war.
	\item[colorback=title] Der Titel (ohne Untertitel) wird hinterlegt.
	\item[colorback=head] Der komplette Titelblock, inklusive Titelblock wird hinterlegt.
	\item[colorback=body] Es wird nur der Bildbereich eingefärbt.
\end{description}

\subsection{Strukturierungselemente}
\minisec{Die abstract-Umgebung}
Die \code{abstract}-Umgebung wird für \cls{tudapub} um eine Option für die Sprache erweitert. Somit ist es möglich, mehrere Zusammenfassungen in einem Dokument zu nutzen.

\begin{verbatim}
\begin{abstract}
	Zusammenfassung entsprechend der Dokumentensprache. In diesem Fall Deutsch.
\end{abstract}

\begin{abstract}[english]
	Zusätzliche Zusammenfassung in englischer Sprache
\end{abstract}
\end{verbatim}

Für die Verwendung ist wichtig, dass alle im Dokument genutzten Sprachen geladen werden. Im Falle des Beispiels muss also sowohl \code{ngerman} als auch \code{english} an das \pck{babel}-Paket übergeben werden.


\subsection{PDF/A Konformität}
Die Klasse TUDaPub unterstützt den Standard PDF/A 2b. Der PDF/A-Modus ist automatisch aktiviert. die zugehörige Option kann jedoch über \code{pdfa=false} ausgeschaltet werden.
Nun wird zusätzlich eine \code{\textbackslash.xmpdata}-Datei generiert. Üblicherweise werden die Titeldaten direkt übernommen.

Dies kann jedoch bei der Verwendung einiger Makros innerhalb der Felder zu Problemen führen. Beispielsweise enthält der Titel für dieses Dokument das Makro \code{\LaTeX}. Es können daher nur Textelemente übernommen werden. Ähnlich den Linkbezeichnungen über PDF-Lesezeichen.

Um dieses Problem zu umgehen stellt \cls{tudapub} hierfür das Makro \code{\textbackslash{}Metadata\{\}} zur Verfügung. Hier können sämtliche von \pck{pdfx} verarbeitbaren Variablen nach Schlüssel$=$Wert-Struktur gesetzt werden. Es ist zu beachten, dass dieses Makro nur dann funktioniert, wenn die \code{pdfa}-Ausgabe aktiviert ist. Ist dies nicht der Fall, so gibt \cls{tudapub} eine entsprechende Fehlermeldung zu diesem Widerspruch aus.
Zum Beispiel:
\begin{verbatim}
\Metadata{
	author=Marei Peischl (peiTeX),
	title=LaTeX im Corporate Design der TU Darmstadt,
}
\end{verbatim}
Das Feld \code{publisher} ist mit \enquote{TU Darmstadt} vorbelegt, kann aber überschrieben werden.

Um mehrere Einträge zu trennen, wird das Makro \code{\textbackslash{}sep} genutzt.
\begin{verbatim}
keywords={TU Darmstadt \sep Corporate Design \sep LaTeX}
\end{verbatim}
Wenn der Eintrag selbst Kommata enthalten könnte, dann ist eine Gruppierung um den Eintrag notwendig. Sonst wird der Text nach dem Komma als nächstes Keyword interpretiert.

\minisec{Mögliche Probleme mit älteren Systemen:}

Bei älteren \TeX-Distributionen kann es vorkommen, dass die Farbprofile nicht vorinstalliert sind. In diesem Fall wird eine Fehlermeldung im folgenden Sinn erzeugt:
\begin{verbatim}
No color profile found to use for RGB screen colors
\end{verbatim}
Um diesen Fehler zu beheben, können die notwendigen *.icc-Dateien unter \url{http://mirror.ctan.org/support/colorprofiles} heruntergeladen und entweder installiert oder im Projektordner mit abgelegt werden. Die einfachste Lösung bleibt jedoch, das eigene \TeX-System zu aktualisieren.

Darüber hinaus werden bei \hologo{XeLaTeX} einige Features nicht unterstützt. In diesem Fall erzeugt \code{pdfa=""false} ein kompilierbares Dokument, allerdings sollte für validierbare PDF/A-Dateien auf \hologo{LuaLaTeX} einer möglichst aktuellen Version umgestiegen werden.

\subsection{Zusätzliche Metadaten nach Wunsch der Universitätsbibliothek}
\label{sec:IMRAD}
Es existiert ein Mechanismus um das Strukturierungsmodell IMRaD \cite{imrad} zu kennzeichnen.
Im Stil der einzelnen Teilbereiche können so, über Aufruf von

\begin{verbatim}
\IMRADlabel{introduction}
\IMRADlabel{methods}
\IMRADlabel{results}
\IMRADlabel{discussion}
\end{verbatim}
entsprechende Labels generiert werden. Sie haben den Namen \code{IMRAD:<Schlüssel>}.

Der Prüfmechanismus ist auf Wunsch der Bibliothek standardmäßig aktiviert, kann jedoch durch die Option \code{IMRAD=false} deaktiviert werden.

\subsection{Farbdarstellung}
\label{sec:pdfa-color}
Der PDF/A-Modus konvertiert automatisch eingegebene CMYK-Farben in RGB. Da es keine eindeutige Umwandlung gibt, sollten daher im Modus \code{pdfa=true} keine CMYK-Elemente verwendet werden.

Für die Druckausgabe ist dieser Modus ungeeignet. Ab Version 3.12 gibt \cls{tudapub} eine entsprechende Infomeldung aus, um über diese Umwandlung zu informieren.

\section{Erweiterte Konfigurationsmöglichkeiten}

\subsection{Auswahl des Dokumentenfarbmodus}
Farben unterliegen je nachdem auf welchem Medium sie ausgegeben werden einer anderen Farbmischung. Für die Verwendung bei der Dokumentenerzeugung ist daher wichtig welches Ausgabemedium primär genutzt werden soll.

Technisch zeigt sich dieser Unterschied in Farbmodellen. TUDa-CI unterstützt entsprechend der Richtlinien sowohl ein Farbmodell für die Druckausgabe (cmyk) und für die Bildschirmdarstellung (RGB). Die Umsetzung geschieht über das Paket \pck{xcolor} wobei die entsprechenden Farbwerte für beide Modelle über \pck{tudacolors} hinterlegt sind.

Normalerweise wählt TUDa-CI automatisch ein passendes Modell. Allerdings sorgt die Voreinstellung von \code{pdfa-true} für eine Konvertierung zu RGB (vgl.~\ref{sec:pdfa-color}).

Soll abseits dieser automatischen Abweichung ein bestimmter Farbmodus erzeugt werden, können die \pck{xcolor}-Optionen \code{cmyk} oder \code{RGB} direkt an \cls{tudapub} übergeben werden. Sie werden entsprechend an das Paket weitergereicht und entsprechend der \pck{xcolor}-Dokumentation verarbeitet.

\subsection{Anpassungen, die von den Corporate Design Richtlinien abweichen}

\subsubsection{Schriftgröße}
\cls{tudapub} kann entgegen der Corporate Design Richtlinien auch andere Schriftgrößen verarbeiten. Hierfür wird die \code{fontsize}-Option aus \KOMAScript{} genutzt (z.\,B. \code{fontsize=11pt}). Sofern keine spezielle Schriftgrößendatei für TUDa-CI vorliegt, wird die mit \KOMAScript{} ausgelieferte Datei gewählt.

Beispiele für Abweichungen aus typografischen Gründen sind Beispielsweise auch in den Demo-Dateien für Abschlussarbeiten gezeigt.

\subsubsection{Seitenränder}
Die Zeilenlängen sind laut Corporate Design aus typografischer Sicht zu lang.
Daher existiert die Klassenoption \code{custommargins}. Sie verfügt über die Werte \code{true}, \code{false} und \code{geometry} mit folgender Bedeutung:

\begin{description}
\item[custommargins=false] Standardeinstellung von \cls{tudapub}. Die Ränder entsprechen den Vorgaben des Corporate Design Guidelines. Die Einstellung wird durch \pck{geometry} durchgeführt. Eigene Anpassungen werden durch das Ausführen von \code{\textbackslash{}maketitle} überschrieben.
\item[custommargins=true] Die Einstellungen des Corporate Design Guidelines werden nicht aktiviert. \pck{geometry} wird nicht geladen. Dieser Modus entspricht der Standardeinstellung von \KOMAScript{}. Dadurch werden die Ränder nicht fest definiert, sondern auf Basis des \pck{typearea}-Paketes berechnet \cite[vgl.][]{scrguide}.
\item[custommargins=geometry]  Diese Variante wurde auf Wunsch zur Verfügung gestellt, allerdings wird darauf hingewiesen, dass manuelle Randeinstellungen oft nicht zu einem harmonischen Satzspiegel führt.
\pck{geometry} wird, wie bei \code{false} geladen und vorkonfiguriert. Es ist allerdings möglich kleinere Anpassung durch die Verwendung des Makros \code{\textbackslash{}geometry} zu setzen. Die Einstellungen, die zu Beginn des Dokuments gelten werden gespeichert und nach der Titelseite wiederhergestellt.

Hierbei ist zu beachten, dass die Einstellungen als Ausgangspunkt den Voreingestellten Satzspiegel nutzen (je nach Option mit Randnotizspalte oder ohne). Es ist möglich diese Optionen vor den eigenen mit zurückzusetzen:
\begin{verbatim}
	\geometry{
	   reset,
	   <Eigene Anpassungen>
	}
\end{verbatim}
Die gilt insbesondere für die Optionen \code{includehead}, \code{includefoot} und \code{includemp}.
\end{description}

\minisec{Hinweis zu den Kopf-/Fußzeilen}
Wenn die Option \code{marginpar=true} gesetzt bleibt, ragen die Kopf- und Fußzeile über die Marginalspalte hinaus. Aus ästhetischen Gründen wird daher empfohlen in diesem Fall die Kopf- und Fußzeile  mit \code{marginpar=false}  auf den Textbereich zu beschränken.

Auch ist das Standard-Layout der Kolumnentitel wenig vorteilhaft, da die Kolumnentitel damit lokal größer sein können als die eigentliche Überschrift. (\code{headline=automark})
Deswegen kann über
\begin{verbatim}
	\pagestyle{TUDa.headings}
\end{verbatim}
ein einfacherer Seitenstil ausgewählt werden, der die Nutzung mit lebenden Kolumnentitel erheblich vereinfacht. Dieser Stil ist über \pck{scrlayer-scrpage} realisiert und kann entsprechend der \KOMAScript{}-Dokumentation angepasst werden.

\minisec{Hinweis zur Bindekorrektur}
Bei Verwendung einer Bindekorrektur (\code{BCOR=<Länge>}) wird diese nicht automatisch auch auf der Titelseite eingefügt. Für diesen Fall wurde mit Version 3.0 zusätzlich die Option \code{BCORtitlepage} hinzugefügt. Falls diese aktiviert wird, nimmt die Titelseite den Wert der Typearea Option \code{BCOR} auf der ersten Seite als Zusatz zum linken Rand hinzu.


\subsection{Frontmatter/Mainmatter/Backmatter}
Üblicherweise existieren die Makros \verb+\frontmatter+, \verb+\mainmatter+ und \verb+\backmatter+ lediglich bei der Basisklasse \cls{scrbook}.
Auf Wunsch wurden diese Makros auch bei \cls{scrartcl} und \cls{scrreprt} als Basis bereitgestellt.

Somit ist es möglich, für den Vorspann auf römische Ziffern zu wechseln. Ab \verb+\mainmatter+ werden dann arabische Ziffern verwendet.


\subsection{Mathematikschriften}
Da es keine Compiler-unanbhängige, universelle Mathematikschrift gibt und die Corporate Design Richtlinien auch keinerlei Empfehlung berücksichtigen, wurden hierfür einige mögliche Varianten diskutiert. Die Voreinstallung entspricht immer dem Standard der Installation. Es werden keine spezifischen Einstellungen geladen.

Die Diskussion hierzu findet sich unter:\\
\url{https://github.com/tudace/tuda_latex_templates/issues/19}

Im Folgenden werden ein paar Beispielkonfigurationen gezeigt. Grundsätzlich ist die Mathematikschriftart jedoch -- abgesehen durch Einschränkungen des Compilers -- frei wählbar.

Bei Auswahl und Verwendung ist häufig der \enquote{\TeX{} Font Catalogue hilfreich}: \url{https://tug.org/FontCatalogue/mathfonts.html}

\subsection{\hologo{pdfLaTeX}}
Hier existiert eine Variante, die die Buchstaben der Basisschriftart \enquote{Charter} mit Mathematiksymbolen aus unterschiedlichen Zeichensätzen möglichst passend kombiniert.

\begin{verbatim}
\usepackage[charter,expert]{mathdesign}
\end{verbatim}

Es gibt ähnliche Ansätze für ein paar weitere Kombinationen. Einige Beispiele finden sich in der XCharter Dokumentation. \url{http://mirrors.ctan.org/fonts/xcharter/doc/xcharter-doc.pdf}


\section{Fachbereichsspezifische Anpassungen}
Einige Fachbereiche haben spezielle Anforderungen. Dieser Abschnitt betrachtet die speziellen Anpassungen. Bisher (Juli 2020) existieren diese Modifikationen lediglich für den Fachbereich Maschinenbau, der Mechanismus ist jedoch erweiterungsfähig.


\subsection{Fachbereich Maschinenbau}
Der entsprechende Modus wird über die Option \code{department=mecheng} oder kurz \code{mecheng} aktiviert. Die Farbgebung passt sich dann automatisch an und die Trennlinie der Fußzeile erhält die geforderte Form des Zeitstrahles. 

Darüber hinaus erfordern manche Stellen dieses Fachbereichs die Übergabe einer ID zu Kennzeichnung des Dokuments. Hierfür wurde der Mechanismus
\begin{verbatim}
\SetPaperID{<Buchstabe>}{<tiefgestelle Nummer>}}
\end{verbatim}
eingeführt.
Dieser funktioniert auch ohne die Aktivierung des \code{mecheng}-Modus, allerdings ergänzt die Option einige zusätzliche Parameter für angepasste Abstände.

Darüber hinaus aktiviert der Modus die Optionen:
\code{colorback=false}, \code{ruledheaders=section}.

\minisec{Logo}
Das Fachbereichslogo wird über die Option \verb+departmentlogofile=tuda_maschinenbau_logo+ übergeben. Über diese Option kann auch eine abweichende Datei genutzt werden. Falls der Wert leer bleibt, wird kein Bild eingefügt.

\minisec{Farbanpassungen}
Der Fachbereich untergliedert die im Corporate Design Handbuch beschriebenen Farben zusätzlich. Daher existieren, wenn \code{mecheng} aktiviert wurde zusätzlich die folgenden Farbnamen:

\begin{verbatim}
\colorlet{TUDa-Primary1}{TUDa-6b}
\colorlet{TUDa-Primary2}{TUDa-2d}
\colorlet{TUDa-Secondary1}{TUDa-9a}
\colorlet{TUDa-Secondary2}{TUDa-8a}
\colorlet{TUDa-Secondary3}{TUDa-6a}
\colorlet{TUDa-Secondary4}{TUDa-3a}
\colorlet{TUDa-Secondary5}{TUDa-4a}
\colorlet{TUDa-Secondary6}{TUDa-5a}
\colorlet{TUDa-Arrow}{TUDa-Primary2}
\end{verbatim}

\minisec{Zeitstrahl}
Des Design-Element des Zeitstrahls kann über das Makro \verb+\MechEngArrow{<Länge>}+ erzeugt werden. Die Farbe entspricht dabei der Farbe \code{TUDaArrow}, die mit der zweiten Primärfarbe (blau) vorbelegt ist.

\section{Weitere Konfigurationsmöglichkeiten: Standard-KOMA-Script}
\label{sec:KOMA}
Da die Klasse bis auf ein paar erzwungene Einstellungen, die das Layout betreffen, vollständig \KOMAScript"=kompatibel ist, ist für sämtliche Modifikationen ein Blick in die \KOMAScript-Dokumentation hilfreich. Für einen Großteil bietet \KOMAScript{} eine eigene Lösung, wodurch Ergänzungspakete oft hinfällig sind.

Beispiele für typische Modifikationen, die auch im Rahmen des Corporate Design zulässig sind:
\begin{itemize}
	\item Umstellung der Absatzkennzeichnungsmethode von Einzug auf Abstand (Klassenoption \verb+parskip+)
	\item Elementnummerierung mit oder ohne Endpunkt (Klassenoption \verb+numbers=enddot/noenddot+)
	\item Positionierung, Ausrichtung und Abstände bei captions (Die Makros \verb+\captionsabove+, \verb+\captionbelow+, \verb+\captionof+ und die Option \verb+captions+)
\end{itemize}



\section{Bekannte Probleme}

\subsection{\texorpdfstring{\hologo{XeLaTeX}}{XeLaTeX} und PDF/A}
Das Paket \pck{pdfx}, über welches die PDF/A Kompatibilität erzeugt wird, hat nur einen begrenzten Support für \hologo{XeLaTeX}.
Es wird eine entsprechende Warnung erzeugt, allerdings kann es bei älteren \hologo{XeLaTeX}-Versionen passieren, dass \pck{pdfx} bereits Fehlermeldungen erzeugt. Abhilfe wird in diesem Fall durch eine Compiler-Wechsel auf \hologo{LuaLaTeX} (welcher ohnehin empfohlen wird) oder durch Abschalten des PDF/A-Modus (\code{pdfa=false})  geschaffen.

\subsection{DVI-Ausgabe}
Aufgrund der Voreinstellung zur Erzeugung einer PDF/A-Datei ist es nicht möglich TUDa-CI in Standardeinstellungen zur Erzeugung einer DVI-Datei zu nutzen. Ein Großteil der Funktionalität kann jedoch durch Deaktivierung der \code{pdfa}-Option genutzt werden.


\subsection{Geschachtelte Akzente (Inkompatibilität zwischen pdfx und amsmath)}
Es ist nach aktuellem Paketstatus (v3.04) nicht möglich beide Pakete in Kombination ohne weitere Anpassungen so zu verwenden, dass eine Schachtelung von Akzenten möglich ist. Das folgende Beispiel wirft Fehler, sobald \pck{amsmath} direkt oder indirekt geladen wird.
\begin{verbatim}
$\dot{\hat{x}}$
\end{verbatim}

\noindent Die zugehörige Diskussion findet sich im tuda-ci Repository (\url{https://github.com/tudace/tuda_latex_templates/issues/78}). Da sich diese Problematik nicht direkt aus tuda-ci sondern aus der Konstruktion der beiden Pakete untereinander ergibt, ist es leider nicht möglich pauschal eine Anpassung in tuda-ci zu implementieren. Unter dem angegebenen Link findet sich jedoch ein möglicher Workaround, sobald beide Varianten benötigt werden.

\subsection{Möglicher Option Clash mit microtype}
Das Paket \pck{microtype} wird im Falle der Nutzung von \hologo{pdfLaTeX} oder einer erzwungenen Nutzung von Type1 Schriftarten automatisch geladen, da in der verwendeten Schriftart die Ligaturen für Kapitälchen deaktiviert werden müssen, um Probleme zu vermeiden (vgl. \url{https://github.com/tudace/tuda_latex_templates/issues/144}).

Falls es zwingend Notwendig ist die Type 1 Schriftart zu verwenden, ist es möglich weitere Optionen vor dem Laden der Dokumentenklasse an \pck{microtype} zu übergeben.

\begin{verbatim}
	\PassOptionsToPackage{<microtype Optionen>}{microtype}
	\documentclass{…}
\end{verbatim}
\appendix
\printbibliography

\end{document}
