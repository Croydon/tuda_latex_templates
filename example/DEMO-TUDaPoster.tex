%% This is file `DEMO-TUDaPoster.tex' version 4.00-beta (2023-10-19),
%% it is part of
%% TUDa-CI -- Corporate Design for TU Darmstadt
%% ----------------------------------------------------------------------------
%%
%%  Copyright (C) 2018--2023 by Marei Peischl <marei@peitex.de>
%%
%% ============================================================================
%% This work may be distributed and/or modified under the
%% conditions of the LaTeX Project Public License, either version 1.3c
%% of this license or (at your option) any later version.
%% The latest version of this license is in
%% http://www.latex-project.org/lppl.txt
%% and version 1.3c or later is part of all distributions of LaTeX
%% version 2008/05/04 or later.
%%
%% This work has the LPPL maintenance status `maintained'.
%%
%% The Current Maintainers of this work are
%%   Marei Peischl <tuda-ci@peitex.de>
%%   Markus Lazanowski <latex@ce.tu-darmstadt.de>
%%
%% The development respository can be found at
%% https://github.com/tudace/tuda_latex_templates
%% Please use the issue tracker for feedback!
%%
%% If you need a compiled version of this document, have a look at
%% http://mirror.ctan.org/macros/latex/contrib/tuda-ci/doc
%% or at the documentation directory of this package (if installed)
%% <path to your LaTeX distribution>/doc/latex/tuda-ci
%% ============================================================================
%%
% !TeX program = lualatex
%%

\documentclass[
	paper=a0,
	ngerman,
	accentcolor=9c,
	logo=body,% Logo unterhalb der Identitätsleise
	footer=true,
%	logofile=example-image, %Falls die Logo Dateien nicht vorliegen
	]{tudaposter}

\usepackage[english, main=ngerman]{babel}
\usepackage[autostyle]{csquotes}


%Formatierungen für Beispiele in diesem Dokument. Im Allgemeinen nicht notwendig!
\let\file\texttt
\let\code\texttt
\let\tbs\textbackslash

\usepackage{multicol}

\begin{document}

\title{\LaTeX{} im Corporate Design der TU~Darmstadt}
\subtitle{Die Dokumentenklasse tudaposter}
\author{Marei Peischl\thanks{pei\TeX{} \TeX{}nical Solutions}\and der \TeX-Löwe}

\titlegraphic{\color{red!20}\rule{\contentwidth}{.3\contentheight}}

\addTitleBox{test}
\addTitleBox{jaodsijf}
\footerqrcode{https://peitex.de}
\footer{Inhalt der Fußzeile}%Falls aktiviert


\maketitle

\section*{Grundlengende Informationen}
Die Dokumentenklasse tudaposter dient der Erstellung von Aushängen und nicht-wissenschaftlichen-Plakaten im Stil der TU-Darmstadt. Sie ist Teil des TUDa-CI-Bundles.

\begin{multicols}{2}
\subsection*{Verwendung}
Im wesentlichen entspricht Ihre Verwendung der Klasse tudapub, da die Titel ähnlich aufgebaut sind. Unterhalb des Titels, kann der Anwender normalen Fließtext schreiben oder wie bei Standard-\LaTeX{} formatieren.

Die tudaposter-Klasse basiert wie auch tudapub auf \KOMAScript{} und bietet daher mehr Mechanismen als für Poster grunsätzlich notwendig sind.

Im Unterschied zur tudasciposter-Klasse, die auf tikzposter basiert, ermöglicht tudaposter es wie gewohnt Fließtext zu schreiben, um ergänzende Informationen einfach zu platzieren.

\subsection*{Titelei}
Die Titelerzeugung funktioniert wie bei Standard-\LaTeX{} über den maketitle-Mechanismus. Neben der Makros title, subtitle und author stehen noch
titlegraphic, addTitleBox, footerqrcode und footer zur Verfügung. Bis auf footerqrcode wird lediglich der Inhalt entsprechend platziert. Ein Beispiel für die Verwendung ist der Datei DEMO-TUDaPoster.tex gezeigt.

\subsection*{Längenangaben}
Innerhalb des Posterinhaltes (Dazu zählt auch die titlegraphic) sind zwei Längenmaße vordefiniert.
\texttt{\textbackslash{}contentwidth} und \texttt{\textbackslash{}contentheight}. Die Höhe entspricht dabei dem Abstand zwischen Titelblock und Fußzeile/Trennlinie.

\subsection*{Dokumentenklassenoptionen}
\begin{description}
	\item[paper=<Papierformat>] Papierformat. Voreingestellt ist a0. Unterstützt werden Formate von A0 bis A4.
	\item[fontsize=<Schriftgröße/auto>] Basischriftgröße. Die anderen Größen werden entsprechend skaliert. Für die unterstützte Papierformate sind entsprechende Skalierungsschritte hinterlegt, diese Voreinstellung entspricht dem Wert auto.
	\item[logo=head/body] Das Logo wird in der Idenditätsleiste (head) oder innerhalb des Dokumenteninhalts, ggf. überlappend mit dem Bild gesetzt.
	\item[color=<Farbkürzel>] Akzentfarbe nach den Corporate Design Richtlinien.
	\item[colorsubtitle=true/false] Hintergrundfarbe beim Untertitel an-/ausschalten.
	\item[footer=true/false] Aktivierung/Deaktivierung der Fußzeile. Voreinstellung ist false.
	\item[marginpar=true/false] Ermöglicht die Nutzung einer Randnotizspalte . Diese hat die Breite des Logos. Sie wird entweder über das Makro \code{\tbs{}SetMarginpar} oder über das Standard-Makro \code{\tbs{}marginpar} befüllt. Eine Verwendung dieses Modus ist in der zusätzlichen Demo-Datei \file{DEMO-TUDaAnnouncement} gezeigt.

	Die Randnotizspalte überlappt die Abbildung, die zum Titel gehört. Kombinationen von \code{marginpar=true} und \code{\tbs{}titlegraphic} sollten daher nur mit Vorsicht genuzt werden.

	\item[title=large/small/default] Ermöglicht es die Schriftgrößen um eine Stufe zu verkleinern. Die Option \code{large} entspricht hierbei der Option \code{default}. Im Fall des Wertes \code{small} werden die Titelschriftgrößen des nächsten kleineren Papierformates verwendet. Die Basischriftgröße und die Randeinstellungen bleiben davon unberührt.
	\item[type=default/announcement] Diese Option liefert die Möglichkeit bestimmte Posterlayouts über einen Einzigen Wert zu laden.
	Der Wert \code{announcement} setzt die folgenden Optionen: \code{marginpar=true, indenttext=false, logo=head, title=small,colorsubtitle=true} und aktiviert zusätzlich die Ausgabe des Untertitels in fetter Schrift.
\end{description}
\end{multicols}

\end{document}
